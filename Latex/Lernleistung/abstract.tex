\begin{abstract}
\jftree
\textbf{AuVi} bezeichnet eine Gruppe von verschiedenen Projekten unter der Leitung von Ronald Eixmann.
Ein zentrales Projekt gibt dieser Arbeit ihren Namen.
AuVi steht für Automatisierte Visualisierung von meteorologischen Daten.
Dieses ermöglicht einem Nutzer über ein benutzerfreundliches Interface oder App
eine Wetterprognose für jeden beliebigen Punkt auf der Erde mit über 50 verschiedenen Parametern abzufragen.
Diese Prognose kann durch das OpenDap System bis zu zehn Tage in die Zukunft abgegeben werden,
weitere Quellen können vom Benutzer hinzugefügt werden um den Parameterumfang zu erweitern.\\
Nach der Teilnahme am \jf Wettbewerb wurde die Projektarbeit um eine Partnerschaft mit Makanya erweitert.
So entstand Makanya.com, eine Website um einen Austausch
zwischen Schülern aus Deutschland und Tansania zu ermöglichen.\\
Der letzte Teil des Projektes ist das Weather Monitoring System rund um Kühlungsborn.
Dieses umfasst eine Reihe von Bildschirmen die Wetterdaten sowie aktuelle Informationen anzeigen.
Gleichzeitig wird das System allerdings auch von der Schule genutzt um Informationen im Foyer anzuzeigen.
Für diese Anzeige werden weiterhin auch noch Grafiken automatisiert erzeugt.\\
All diese Teilprojekte sind auch mit Öffentlichkeitsarbeit verbunden.
\end{abstract}
