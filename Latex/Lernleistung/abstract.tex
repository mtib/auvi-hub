\begin{abstract}
\jftree
Mit \textbf{AuVi} bezeichnen wir eine Gruppe von verschiedenen Projekten unter der Leitung von Ronald Eixmann.
Die Projektarbeit findet von Markus Becker und Swenja Wagner statt.
Unser zentrales Projekt gibt unserer Arbeit ihren Namen. AuVi steht für Automatisierte Visualisierung von meteorologischen Daten, dem Projekt mit dem wir an verschiedensten \jf Wettbewerben teilnahmen.
Dieses ermöglicht einem Nutzer über eine Website oder App eine Wetterprognose für jeden beliebigen Punkt auf der Erde mit über 50 verschiedenen Parametern abzufragen.
Diese Prognose kann bis zu zehn Tage in die Zukunft abgegeben werden.\\
Später wurde unsere Projektarbeit um eine Partnerschaft mit Makanya erweitert.
So entstand Makanya.com, eine Website um einen Austausch zwischen Schülern aus Deutschland und Tansania zu ermöglichen.\\
Der letzte teil unseres Projektes ist das Weather Monitoring System rund um Kühlungsborn.
Dieses umfasst eine Reihe von Bildschirmen die Wetterdaten sowie aktuelle Informationen anzeigen.
Gleichzeitig wird das System allerdings auch von der Schule genutzt um Informationen im Foyer anzuzeigen.\\
All diese Teilprojekte sind natürlich auch mit Öffentlichkeitsarbeit verbunden.
\end{abstract}
