% NOTE Großteil des Quellcodes muss in den Anhang
\newcommand{\ghl}[1]{\link{http://github.com/mtib/#1}}	% Format Links
\newcommand{\ghx}[1]{\link{http://github.com/eixmb/#1}}	% Format Links
\newcommand{\img}[2]{\begin{center}#2\\\includegraphics[width=\linewidth]{imgs/#1}\end{center}}
\newpage
\section{Anhang}
Der komplette aktualisierte Quelltext ist für AuVi / \vs\, unter \link{github.com/mtib/auvi-hub} und für das Weather-Monitoring-System unter \link{github.com/mtib/apewms} hochgeladen und frei verfügbar. Der grafische Teil des Weather-Monitoring-Systems und die Python API ist unter \link{github.com/mtib/autowkid} veröffentlicht.\\
Die Open-Source Teile des in Golang geschriebenen Systemes sind unter \\\ghx{aktuelltxt} und \ghl{simplehttp} zu finden.\\
\img{jf.png}{
Projektstruktur:}
\newpage
\img{pi.png}{
Raspberry Pi 2, Hardware auf der alle Software getestet wird.}
\img{gl/ps_0001.jpg}{
Globaler Luftdruck:}
\newpage
\img{gl/tql_0021.jpg}{
Globale Wassersäule:}
\img{ro/t2m_tql_last.jpg}{
Funktionsgraph Temperatur und Regenwahrscheinlichkeit:}
\newpage
\img{eu/ps_0033.jpg}{
Luftdruck um Europa:}
\img{eu/t2m_0023.jpg}{
Temperatur um Europa:}
\newpage



\img{makanyaFrontEnd.png}{
Makanya Frontend für Hompage und soziales Netzwerk}
