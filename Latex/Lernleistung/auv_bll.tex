
\documentclass[a4paper,oneside,10pt,titlepage]{article}

\usepackage[ngerman]{babel}
\usepackage[utf8]{inputenc}
\usepackage{a4wide}
\usepackage{lmodern}
\usepackage{graphicx}
\usepackage{amsmath}
\usepackage{amsfonts}
\usepackage{eurosym}
\renewcommand*{\familydefault}{\sfdefault}
%Test von Swenja

\begin{document}
\linespread{.9}
\pagestyle{empty}
\title{Personalisierbare\\Datenverarbeitung und Visualisierung\\von Wetterprognosedaten}
\author{\\\\\\\\\\\\\\\\\\\\\\Markus Becker\\Swenja Wagner\\Tobias Gerding}

\maketitle
\pagestyle{empty}
\tableofcontents
\thispagestyle{empty}
\pagestyle{plain}
\newpage

Grafische und nichtgrafische Visualiserung von Wetterprognosedaten. Vom Nutzer können Parameter, Form, Ort, Quelle und weitere meteorologische sowie informatorische Werte bestimmt werden.  
\section{Prognosetheorie}
Was ist eine Prognose in der Meteorologie?
\subsection{Prognosen per Hand}
Was muss man über die aktuelle Situation wissen um per Hand eine Prognose zu erzeugen?
\subsection{Voraussetzung für Datenprognose}
Welche Rechenleistung braucht man um eine Datenprognose am Computer zu erstellen? Antwort: Sehr viel!
\subsection{Quellenvergleich}
Da wir nun dargestellt haben, dass es für uns nicht möglich ist unsere eigenen Daten zu generieren, zeigen wir nun welche Quellen die benötigten Daten zur Verfügung stellen. Der Nutzer kann selbst zwischen den verschiedenen Datenquellen wählen, die angeboten werden. Die angebotenen Datenquellen können über die GUI gewählt werden.

Wir werden über eine GUI (graphic user interface) dem Nutzer anbieten zwischen verschiedenen Datenquellen zu wählen. Diese werden wir auch statistisch vergleichen.
Grafik: Übersicht über Quellen (Alle die direkt über das Programm angeboten werden)
Programm kann um weitere erweitert werden.
Auswahl NOAA, GFS. 
\subsubsection{Deutscher Wetter Dienst}
Wir geben an was der DWD bietet und wie viel er kostet.
Grafik: Diagramm der Genauigkeit
\subsubsection{NOAA Global Forecast System}
Wir geben an was das GFS bietet und warum wir es als Hauptdatenquelle verwenden.
Grafik: Diagramm der Genauigkeit

Das Global Forecast System (GFS) ist ein Modell, das mathematisch Parameter errechnet. Die dazu benötigten Daten bezieht es aus einem Netz von Wetterstationen, die sowohl an Land als auch im Wasser und in der Luft Messungen durchführen. Mit diesen gemessenen Daten werden mithilfe von geophysikalischen Gesetzen weitere Daten errechnet. So werden für ein 13,5km-maschiges Raster pro Schnittpunkt je ein Wert ermittelt. Diese Daten werden als große Datensätze kostenfrei auf der Webseite ()  zur Verfügung gestellt.
\subsection{Datenverfügbarkeit}
Wie weit in die Zukunft können die Quellen schauen. Wodurch sind sie begrenzt, Wie oft muss man die Daten generieren und abrufen?
Grafik: Diagramm der Zukunftsdaten
\section{AuVi-Programm}
Wofür steht AuVi?
\subsection{Spezifikation}
Was muss das Programm können?
Personalisierte Anfragen auf Prognosen lokal umsetzen. Zum Erzeugen eine Tabelle. Der Nutzer legt von Skala bis Parameter und Hintergrundfarbe alles fest.
\subsection{Programmablauf}
Was macht das Programm?
\subsection{Programmierung}
Wie wurde das Programm umgesetzt? Welche Voraussetzungen müssen gegeben sein?
\subsection{Versionskatalog}
Welche Versionen gibt es von diesem Programm (Welche Sprachen)?
\subsection{Fehlerbehebung}
Was wurde getan um dafür zu sorgen, dass das Programm funktioniert?
\section{Technik}
% Nutzer generiert eigene Ergebnisse
% Server ist der von uns öffentliche Server zum verteilen des Systems
Auf welcher Hardware soll das Programm in welcher Version laufen. Es wird eine Server Version geben, die die Daten für einen Webserver bereitstellt, und eine die nur für lokale Anzeigen gedacht ist. An diese Rechner gibt es andere Anforderungen.
\subsection{Server-Voraussetzungen}
Internet, Speicher,
\subsection{Nutzer-Voraussetzungen}
Internet, Bildschirm, Interpreter
\subsection{Generations-Voraussetzungen}
AuVi, Internet, Speicher
\section{Software}
Welche Software wird/wurde zur Entwicklung genutzt?
\subsection{Alte Abhängigkeiten}
Ältere Versionen hatten mehr Abhängigkeiten. Neuere nicht.
\subsection{Dateiformate}
Welche Datenausgabeformate unterstützen wir?
\subsection{Anzeigeformate}
Anzeige auf beliebige Geräte auf dem größten Nenner.
\subsubsection{Programmvergleich}
Java und Python Programme auf Nutzer- und Generatorenseite.
\subsubsection{Webbrowservergleich}
Damit jeder die offizielle AuVi Seite aufrufen kann ist diese im Web mit Beispielbildern. Diese lassen sich auf unterschiedlichen Browsern unterschiedlich gut ansehen. Hier der Vergleich.
\section{Nutzen und Nachfrage}
Kurze Anekdote, dass es noch kein Programm gibt, das dem Nutzer diese Freiheit lässt.
\subsection{Interfaces}
Daten sollen erreichbar sein.
\subsubsection{Website}
Einfachste Variante für Nutzer. Als Website mit dynamischem Inhalt
\subsubsection{Vollbildanwendung}
Lokale Anwendung, damit kein Pseudowebserver im Webbrowser laufen muss.
\subsubsection{App}
App um auf Webserver zuzugreifen.
\subsection{Szenarien}
Wir werden nun Beispiele angeben, wie unser Programm genutzt werden kann. Es gibt darüber hinaus unendlich viele Möglichkeiten aus dem Programm individuellen Nutzen zu ziehen.
\subsubsection{Frühwarnung}
Nach dem Ampel System eigene Parameter die Gefährdung darstellen markieren. Z.B. Segler mit Windgeschwindigkeit auf Seelevel, Segelflieger mit Windgeschwindigkeit und Richtung auf x-Meter Höhe. Allgemeiner Arbeiter mit Temperatur.
\subsubsection{Exotische Orte}
Ob mitten im Ozean oder in der Wüste, unser Programm liefert Prognosen für jeden Ort. Genauigkeit von der Zeitspanne und der Entfernung zur physikalischen Messstation abhängig.
\subsubsection{Wissenschaft}
Wissenschaftler können von uns bereitgestellte Parameter kombinieren und sich so neben allgemein komplexeren meteorologischen Daten auch besonders interessante meteorologische Ereignisse grafisch darstellen lassen.
% Dr. Eixmann fragen, was man sich wissenschaftlich visualisieren könnte
Grafik: Ozon/Temperatur o.ä.
\subsubsection{Anpassung}
Wie zuvor genannt kann der Nutzer eigene Muster einstellen und so das Programm erweitern.
Grafik: Tabelle im Programm, Text als Datei.
% Inhalte für die besondere Lerleistung
\section{Arbeitsmethode}
Wer ist in der Projekt, Wer war in dem Projekt?
\subsection{Arbeitsteilung}
Wer hat was gemacht?
\subsection{Bearbeitung der Themengebiete}
Wie hat wer was gemacht?
\section{Aktuelle Anwendungen}
So kann man unser Programm jetzt nutzen, so kann man es erreichen!
\subsection{Hotels, Hafenhaus, Marina, Schule, ...}
Wirkliche Anwendungen nach Muster 7.1.
\section{Quellen}
Auf die Datenquellen wird gesondert eingegangen um sie noch ein mal zu differenzieren.
\subsection{Datenquellen}
Verschiedene Datenquellen mit Anbieter und Link zur Dokumentation.
\subsection{Sonstige}
Alle anderen Quellen, sowie Software.

\newpage
\Large{Selbstständigkeitserklärung}\\
\\
\small Hiermit erklären wir, dass wir die vorliegende Arbeit selbständig angefertigt, nicht anderweitig zu Prüfungszwecken vorgelegt und keine anderen, als die angegebenen Hilfsmittel verwendet haben. Zudem waren alle verwiesenen Webseiten zum Zeitpunkt der Linksetzung gültig und erreichbar. Wörtlich und sinngemäße Übernahmen aus anderen Werken sind als solche gekennzeichnet.
\\ 



\end{document} 
